
% This LaTeX was auto-generated from MATLAB code.
% To make changes, update the MATLAB code and republish this document.

\documentclass{article}
\usepackage{graphicx}
\usepackage{color}

\sloppy
\definecolor{lightgray}{gray}{0.5}
\setlength{\parindent}{0pt}

\begin{document}

    
    \begin{verbatim}
% This programm reads ECG data which are saved in format 212.
% and Respiratory data which are saved in format 16 (apnea-ecg database)
% The annotations are saved in the vector ANNOT, the corresponding
% times (in seconds) are saved in the vector ATRTIME.
% The annotations are saved as numbers, the meaning of the numbers can
% be found in the codetable "ecgcodes.h" available at www.physionet.org.
%
%
%      created on 2015 by
%      Santiago Arevalo (National University of Entre Rios)
%      (email: sarevalog@correo.udistrital.edu.co),
%
%      algorithm is based on a program written by
%      Klaus Rheinberger (University of Innsbruck)
%      (email: klaus.rheinberger@uibk.ac.at)
%
% -------------------------------------------------------------------------
clc; clear all;

PATH= '/home/sarevalog/Dropbox/MAESTRIA/THESIS/CODIGOS/MATLAB/RESPIRATORY';

%Lee la señal ECG de un archivo y las señales respiratorias de otro

sig='a01';

HEADERFILE= [sig 'er' '.hea'];      % header-file in text format

DATAFILEECG=[sig '.dat'];         % data-file

DATAFILERES=[sig 'r' '.dat'];         % data-file

ATRFILE = [sig 'r' '.apn'];       % apnea annotations

SAMPLES2READ = 50;
% ------ LOAD HEADER DATA --------------------------------------------------

fprintf(1,'\\n$> WORKING ON %s ...\n', HEADERFILE);
signalh= fullfile(PATH, HEADERFILE);
fid1=fopen(signalh,'r');
z= fgetl(fid1);
A= sscanf(z, '%*s %d %d %d',[1,3]);
nosig= A(1);  % number of signals
sfreq=A(2);   % sample rate of data
samples = A(3);
clear A;
for k=1:nosig
    z= fgetl(fid1);
    A= sscanf(z, '%*s %d %d %d %d %d %d %d %*s',[1,5]);
    dformat(k)= A(1);           % format;
    gain(k)= A(2);              % number of integers per mV
    bitres(k)= A(3);            % bitresolution
    zerovalue(k)= A(4);         % integer value of ECG zero point
    firstvalue(k)= A(5);        % first integer value of signal (to test for errors)
end;
fclose(fid1);
clear A fid1;
%
%------ LOAD BINARY DATA --------------------------------------------------
%-------
%-------ECG
%-------

signalECG= fullfile(PATH, DATAFILEECG);            % data in format 212
fid2=fopen(signalECG,'r');                         %'r' open file for reading
A= fread(fid2, [2,inf], 'uint8')';                 % matrix with 3 rows, each 8 bits long, = 2*12bit
fclose(fid2);

M1HA= bitand(A(:,2), 15);
PRLA=bitshift(bitand(A(:,2),8),9);     % sign-bit
ECG( : , 1)= bitshift(M1HA,8)+ A(:,1)-PRLA;
if ECG(1,1) ~= firstvalue, error('inconsistency in the first bit values'); end;

clear M1HA PRLA A;

%-------
%-------RES
%-------

signalRES= fullfile(PATH, DATAFILERES);            % data in format 16
fid3=fopen(signalRES,'r');                         %'r' open file for reading
B = fread(fid3,[4,inf], 'uint16')';                % matrix with 4 rows
fclose(fid3);
n=length(B);

RES= zeros(samples, 4);                                  % preallocating

for k=1:n
    val1 = B(k,1);
    val2 = B(k,2);
    val3 = B(k,3);
    val4 = B(k,4);
    y1 = sign(2^(16-1)-val1)*(2^(16-1)-abs(2^(16-1)-val1));
    y2 = sign(2^(16-1)-val2)*(2^(16-1)-abs(2^(16-1)-val2));
    y3 = sign(2^(16-1)-val3)*(2^(16-1)-abs(2^(16-1)-val3));
    y4 = sign(2^(16-1)-val4)*(2^(16-1)-abs(2^(16-1)-val4));

    if ((y1 == 0) && (val1 ~= 0))
    RES(k , 1) = -val1;
    else
    RES(k , 1) = y1;
    end

    if ((y2 == 0) && (val2 ~= 0))
    RES(k , 2) = -val2;
    else
    RES(k , 2) = y2;
    end

    if ((y3 == 0) && (val3 ~= 0))
    RES(k , 3) = -val3;
    else
    RES(k , 3) = y3;
    end

    if ((y4 == 0) && (val4 ~= 0))
    RES(k , 4) = -val4;
    else
    RES(k , 4) = y4;
    end
end;

clear val1 val2 val3 val4 y1 y2 y3 y4 n fid2 fid3;

for k=2:nosig

    RES( : , k -1 )= (RES( : , k -1)- zerovalue(k))/gain(k);

end;
TIME=((0:(samples-1))/sfreq)';
signal =[TIME , ECG , RES];

clear ans bitres B dformat ECG firstvalue gain k nosig RES samples;
clear signalECG signalh signalRES z zerovalue;

fprintf(1,'\\n$> LOADING DATA FINISHED \n');

%------ LOAD ATTRIBUTES DATA ----------------------------------------------
% read from Physionet server

% A=rdann('apnea-ecg/a01er','apn');

atrd= fullfile(PATH, ATRFILE);      % attribute file with annotation data
fid3=fopen(atrd,'r');
A= fread(fid3, [2, inf], 'uint8')';
fclose(fid3);

ANNOT=[];
ATRTIME = [];
sa=size(A);
saa=sa(1);

i=1;
while i <= saa
    annoth = bitshift(A(i,2),-2);
    if annoth == 59
        ANNOT = [ANNOT;bitshift(A(i + 3,2),-2)];
      ATRTIME = [ATRTIME;A(i+2,1) + bitshift(A(i + 2,2),8) + bitshift(A(i + 1,1),16) + bitshift(A(i + 1,2),24)];
        i = i + 3;
    elseif annoth == 60
    elseif annoth == 61
    elseif annoth == 62
    elseif annoth == 63
        hilfe = bitshift(bitand(A(i,2),3),8) + A(i,1);
        hilfe = hilfe + mod(hilfe,2);
        i = i + hilfe/2;
    else
        ATRTIME = [ATRTIME;bitshift(bitand(A(i,2),3),8) + A(i,1)];
        ANNOT = [ANNOT;bitshift(A(i,2),-2)];
   end;
   i = i + 1;
end;

ANNOT(length(ANNOT)) = [];                  % Last Line = EOF (= 0)
ATRTIME(length(ATRTIME)) = [];              % Last Line = EOF
ATRTIME = (cumsum(ATRTIME))/sfreq;
ind = find(ATRTIME <= TIME(end));
ATRTIMED = ATRTIME(ind);
ANNOT = round(ANNOT);
ANNOTD = ANNOT(ind);
apnea = [ATRTIMED , ANNOTD];

%---------DISPLAY DATA ----------------------
figure(1); clf, box on, hold on

liminf=720; limsup=780;

plot(signal(liminf:limsup,1),signal(liminf:limsup,2),'g');   %cambio de resp normal a apnea

xlim([TIME(liminf), TIME(limsup)]);
xlabel('Time / s');
ylabel('ECG Voltaje');


figure(2); clf, box on, hold on

plot(signal(:,1),signal(:,6),'r');

xlim([TIME(1), TIME(end)]);
xlabel('Time / s');
ylabel('SpO2');

figure(3); clf, box on, hold on

liminf=72000; limsup=78000;

plot(signal(liminf:limsup,1),signal(liminf:limsup,2),'g');   %cambio de resp normal a apnea
plot(signal(liminf:limsup,1),signal(liminf:limsup,6),'r');

for k=13:14
    text(ATRTIMED(k),0,num2str(ANNOTD(k)));
end;
xlim([TIME(liminf), TIME(limsup)]);
xlabel('Time / s');
ylabel('SpO2 & ECG');


figure(4); clf, box on, hold on

plot(signal(:,1),signal(:,2),'r');
plot(signal(:,1),signal(:,6),'g');

clear A ANNOT ANNOTD annoth ans ATRTIME ATRTIMED fid3 i ind sa saa sfreq TIME;
fprintf(1,'\\n$> DISPLAYING DATA FINISHED \n');
\end{verbatim}

        \color{lightgray} \begin{verbatim}\n$> WORKING ON a01er.hea ...
\n$> LOADING DATA FINISHED 
\n$> DISPLAYING DATA FINISHED 
\end{verbatim} \color{black}
    
\includegraphics [width=4in]{respiratory_01.eps}

\includegraphics [width=4in]{respiratory_02.eps}

\includegraphics [width=4in]{respiratory_03.eps}

\includegraphics [width=4in]{respiratory_04.eps}



\end{document}
    
